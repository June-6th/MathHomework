\documentclass[../main.tex]{subfiles}
\begin{document}

\subsection{第 4 题}
% 由柯西不等式可得 $A_n$ 为有界线性算子.
任意 $x(t) \in C[-\pi, \pi]$, 设其形式傅立叶级数为
\[
    x(t) = \frac{a_0}{2} + \sum_{i = 1}^{\infty}(a_k \cos{kt} + b_k \cos{kt}),
\]
其前 $n + 1$ 项部分和为:
% ref 书第 68 页
\begin{align}
    \begin{split} \label{eq:1}
         & \frac{a_0}{2} + \sum_{i = 1}^{n}(a_k \cos kt + b_k \sin kt) \\
        =& \frac{1}{\pi} \int_{-\pi}^{\pi} x(s)(\frac{1}{2} + \sum_{i = 1}^{n} \cos k(s - t)) \, ds \\
        =& \frac{1}{2 \pi} \int_{-\pi}^{\pi} x(s) \frac{\sin (n + \frac{1}{2})(s - t)}{\sin \frac{1}{2}(s - t)} \, ds \\
        =& \frac{1}{2} A_n x(t).
    \end{split}
\end{align}
显然, 存在 $x_0(t) \in C[-\pi, \pi]$,
使得其傅立叶级数在 $L^2 [-\pi, \pi]$ 中发散,
也就是其形式傅立叶级数前 $n$ 项和组成的级数 $\{ \frac{1}{2} A_n x(t) \}_{n \in \mathbb{N}}$ 不收敛.
故函数序列 $\{ A_n x(t) \}_{n \in \mathbb{N}}$ 亦发散.
于是, 算子序列 $\{ A_n \}_{n \in \mathbb{N}}$ 在 $x_0(t)$ 处发散,
故其不强收敛也不一致收敛.

下面来看算子序列 $\{ A_n \}_{n \in \mathbb{N}}$ 的弱收敛性.
算子序列 $\{ A_n \}_{n \in \mathbb{N}}$ 在 $x_0$ 处发散,
故存在 $t_0$ 使得数列 $\{ (A_n x_0)(t_0) \}_{n \in \mathbb{N}}$ 发散.
设 $f: A \to (A x_0)(t_0)$, 显然, $f$ 为一线性泛函.
而数列 $\{ f(A_n) \}_{n \in \mathbb{N}} = \{ (A_n x_0)(t_0) \}_{n \in \mathbb{N}}$ 发散,
于是算子序列 $\{ A_n \}_{n \in \mathbb{N}}$ 不弱收敛.

\subsection{第 8 题}
任意 $x(t) \in C[-\pi, \pi] (L^2[-\pi, \pi])$, 设其形式傅立叶级数为
\[
    x(t) = \frac{a_0}{2} + \sum_{i = 1}^{\infty}(a_k \cos{kt} + b_k \cos{kt}),
\]
由等式 (\ref{eq:1}) 可知, 其前 $n + 1$ 项部分和为 $A_n x$.
设所有傅立叶级数收敛的函数组成空间 $X$, 且 $X$ 在 $C[-\pi, \pi]$ ($L^2[-\pi, \pi]$) 中稠密.
容易看出 $A_n$ 在 $C[-\pi, \pi]$ ($L^2[-\pi, \pi]$) 连续, 因此我们只需在 $X$ 上计算 $\| A_n \|$.
对于任意的常值函数 $x$, $A_n x = x$, 故 $\| A_n \| \geqslant 1$.
三角函数为正交基, 故对于任意的 $x \in X$,
在 $L^2[-\pi, \pi]$ 中有 $\| A_n x \| \leqslant \| x \|$, 此时 $\| A_n \| = 1$.
% TODO: 在 $C[-\pi, \pi]$ 中...

\subsection{第 9 题}
设 $d = \sup_{n \in \mathbb{N}} | a_n |$,
$l^2$ 中的点列 $\{ x_n \}_{n \in \mathbb{N}}$, 其中 $x_n$ 的第 $n$ 分量为 $1$, 其他分量为 $0$.
$\| A x_n \| = | a_n |$ 对任意 $n \in \mathbb{N}$ 成立,
故 $d < \infty$ 且 $\| A \| \geqslant d$.
而对于任意的 $x = (x_1, x_2, \cdots, x_n, \cdots)\in l^2$,
\[
    \| A x \| = \left( \sum_{i = 1}^{\infty} a_i x_i \right)^{\frac{1}{2}}
    \leqslant \left( \sum_{i = 1}^{\infty} d x_i \right)^{\frac{1}{2}}
    = d \, \| x \|,
\]
故 $\| A \| \leqslant d$.
综上 $A$ 在 $l^2$ 上连续当前仅当 $\sup_{n \in \mathbb{N}} a_n < \infty$,
且 $\| A \| = \sup_{n \in \mathbb{N}} a_n$.

\subsection{第 10 题}
当 $\lambda \geqslant 1$ 时, $E_{\lambda} x(t) = x(t)$ 对任意的 $x \in L^2[0, 1]$ 成立, 此时 $\| E_{\lambda} \| = 1$.
当 $\lambda \leqslant 0$ 时, $E_{\lambda} x(t) = 0$ 对任意的 $x \in L^2[0, 1]$ 成立, 此时 $\| E_{\lambda} \| = 0$. 
当 $ 0 < \lambda < 1$ 时, 任意 $x(t) \in L^2[0, 1]$
\begin{align*}
    \| E_{\lambda} x(t) \| &= \left( \int_0^1 E_{\lambda} x^2(t) \, dt \right)^{\frac{1}{2}} \\
                           &= \left(\int_0^{\lambda} x^2(t) \, dt \right)^{\frac{1}{2}} \\
                           &\leqslant \left( \int_0^1 x^2(t) \, dt \right)^{\frac{1}{2}},
\end{align*}
故 $\| E_{\lambda} \| \leqslant 1$;
而当 $x(t)$ 满足对任意 $t > \lambda, x(t) = 0$ 时, $E_{\lambda} x(t) = x(t)$, 于是 $\| E_{\lambda} \| \geqslant 1$;
故 $\| E_{\lambda} \| = 1$.
综上, 当 $\lambda \leqslant 0$ 时, $\| E_{\lambda} \| = 0$, 当 $\lambda > 0$ 时 $\| E_{\lambda} \| = 1$.

\end{document}
