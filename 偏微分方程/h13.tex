\documentclass[../main.tex]{subfiles}
\begin{document}

% 习题 6-4
\subsection{第 4 题}
首先 $u = 0$ 确实为方程的解, 下证唯一性.
设 $u$ 为一解, $u$ 在 $\Omega$ 上的最大值点为 $X_1$, 则 $u \left( X_1 \right) \geqslant u \mid_{\partial \Omega} = 0$.
又 $\forall \, i \in \mathbb{N}, 1 \leqslant i \leqslant n$,
\[
    \frac{\partial^2 u}{\partial x_i^2} \leqslant 0
    , \quad
    \frac{\partial u}{\partial x_i} = 0
\]
故 $u \left( X_1 \right)^3 = \Delta_n u \left( X_1 \right) + \left| \bigtriangledown u \left( X_1 \right) \right| \leqslant 0$,
即 $u \left( X_1 \right) \leqslant 0$.
于是 $u \left( X_1 \right) = 0$, 即 $u$ 在 $\Omega$ 上的最大值为 $0$.
容易验证 $- u$ 也是原方程的解, 于是 $- u$ 在 $\Omega$ 上的最大值为 $0$,
此即 $u$ 在 $\Omega$ 上的最小值为 $0$.
综上, $u = 0$.

\subsection{第 10 题}
$\overline{\Omega}$ 为紧集, 故 $u$ 在 $\overline{\Omega}$ 上可取到最大值, 设最大值点为 $x_1$.
首先证明
\[
    u \left( x_1 \right)
    =
    \max_{x \in \overline{\Omega}} u \left( x \right)
    \leqslant
    \max \{ \max_{x \in \partial \Omega} \left| g \left( x \right) \right|, \max_{x \in \Omega} \left| f \left( x \right) \right|^{\frac{1}{2}} \}
\]
$u \left( x_1 \right) \leqslant 0$ 时, 上式显然成立.
$u \left( x_1 \right) > 0$ 时, 若 $x_1 \in \partial \Omega$, 则
\[
    u \left( x_1 \right)
    \leqslant
    \max_{x \in \partial \Omega} \left| g \left( x \right) \right|
    \leqslant
    \max \{ \max_{x \in \partial \Omega} \left| g \left( x \right) \right|, \max_{x \in \Omega} \left| f \left( x \right) \right|^{\frac{1}{2}} \}
\]
若 $x_1 \in \Omega$, 则 $\forall \, i \in \mathbb{N}, 1 \leqslant i \leqslant n$,
\[
    \frac{\partial^2 u}{\partial x_i^2} \leqslant 0
\]
于是
\begin{gather*}
    u \left( x_1 \right)^2
    =
    \left| u \left( x_1 \right) \right| u \left( x_1 \right)
    =
    f \left( x_1 \right) + \Delta_n u \left( x_1 \right)
    \leqslant
    f \left( x_1 \right) \\
    u \left( x_1 \right)
    \leqslant
    \max_{x \in \Omega} \left| f \left( x \right) \right|^{\frac{1}{2}}
    \leqslant
    \max \{ \max_{x \in \partial \Omega} \left| g \left( x \right) \right|, \max_{x \in \Omega} \left| f \left( x \right) \right|^{\frac{1}{2}} \}
\end{gather*}
综上
\[
    u \left( x_1 \right)
    =
    \max_{x \in \overline{\Omega}} u \left( x \right)
    \leqslant
    \max \{ \max_{x \in \partial \Omega} \left| g \left( x \right) \right|, \max_{x \in \Omega} \left| f \left( x \right) \right|^{\frac{1}{2}} \}
\]
观察到 $- u$ 满足以下方程
\[
    \begin{cases}
        - \Delta_n u + \left| u \right| u = - f, & x \in \Omega \\
        u \mid_{\partial \Omega} = - g
    \end{cases}
\]
由上可知
\[
    \max_{x \in \overline{\Omega}} - u \left( x \right)
    \leqslant
    \max \{ \max_{x \in \partial \Omega} \left| - g \left( x \right) \right|, \max_{x \in \Omega} \left| - f \left( x \right) \right|^{\frac{1}{2}} \}
    =
    \max \{ \max_{x \in \partial \Omega} \left| g \left( x \right) \right|, \max_{x \in \Omega} \left| f \left( x \right) \right|^{\frac{1}{2}} \}
\]
综上,
\[
    \max_{x \in \overline{\Omega}} \left| u \left( x \right) \right|
    =
    \max \{ \max_{x \in \overline{\Omega}} - u \left( x \right), \max_{x \in \overline{\Omega}} u \left( x \right) \}
    \leqslant
    \max \{ \max_{x \in \partial \Omega} \left| g \left( x \right) \right|, \max_{x \in \Omega} \left| f \left( x \right) \right|^{\frac{1}{2}} \}
\]

\end{document}
