\documentclass[../main.tex]{subfiles}
\begin{document}

% 书习题 6-1
\subsection{第 1 题}
首先有
\[
    \begin{bmatrix}
        \frac{\partial}{\partial r} \\
        \frac{\partial}{\partial \theta}
    \end{bmatrix}
    =
    \begin{bmatrix}
        \frac{\partial x}{\partial r} & \frac{\partial y}{\partial r} \\
        \frac{\partial x}{\partial \theta} & \frac{\partial y}{\partial \theta}
    \end{bmatrix}
    \begin{bmatrix}
        \frac{\partial}{\partial x} \\
        \frac{\partial}{\partial y}
    \end{bmatrix}
    =
    \begin{bmatrix}
        \cos \theta & \sin \theta \\
        -r \sin \theta & r \cos \theta
    \end{bmatrix}
    \begin{bmatrix}
        \frac{\partial}{\partial x} \\
        \frac{\partial}{\partial y}
    \end{bmatrix}
    .
\]
于是, 在极坐标变换下 $\Delta_2 u$ 的二阶微分部分 $\left( \Delta_2 u \right)_2$ 为
\begin{align*}
    \left( \Delta_2 u \right)_2
    &=
    \begin{bmatrix}
        \frac{\partial}{\partial r} & \frac{\partial}{\partial \theta}
    \end{bmatrix}
    \begin{bmatrix}
        \cos \theta & -r \sin \theta \\
        \sin \theta & r \cos \theta
    \end{bmatrix}^{-1}
    \begin{bmatrix}
        1 & 0 \\
        0 & 1
    \end{bmatrix}
    \begin{bmatrix}
        \cos \theta & \sin \theta \\
        -r \sin \theta & r \cos \theta
    \end{bmatrix}^{-1}
    \begin{bmatrix}
        \frac{\partial}{\partial r} \\
        \frac{\partial}{\partial \theta}
    \end{bmatrix}
    \left( u \right) \\
    &=
    \begin{bmatrix}
        \frac{\partial}{\partial r} & \frac{\partial}{\partial \theta}
    \end{bmatrix}
    \begin{bmatrix}
        1 & 0 \\
        0 & \frac{1}{r}
    \end{bmatrix}
    \begin{bmatrix}
        \cos \theta & \sin \theta \\
        - \sin \theta & \cos \theta
    \end{bmatrix}
    \begin{bmatrix}
        \cos \theta & - \sin \theta \\
        \sin \theta & \cos \theta
    \end{bmatrix}
    \begin{bmatrix}
        1 & 0 \\
        0 & \frac{1}{r}
    \end{bmatrix}
    \begin{bmatrix}
        \frac{\partial}{\partial r} \\
        \frac{\partial}{\partial \theta}
    \end{bmatrix}
    \left( u \right) \\
    &=
    \begin{bmatrix}
        \frac{\partial}{\partial r} & \frac{\partial}{\partial \theta}
    \end{bmatrix}
    \begin{bmatrix}
        1 & 0 \\
        0 & \frac{1}{r^2}
    \end{bmatrix}
    \begin{bmatrix}
        \frac{\partial}{\partial r} \\
        \frac{\partial}{\partial \theta}
    \end{bmatrix}
    \left( u \right) \\
    &= \frac{\partial^2 u}{\partial r^2} + \frac{1}{r^2} \frac{\partial^2 u}{\partial \theta^2}
    ,
\end{align*}
$\Delta_2 u$ 的一阶微分部分 $\left( \Delta_2 u \right)_1$ 为
\begin{align*}
    \left( \Delta_2 u \right)_1
    &= \Delta_2 r \frac{\partial u}{\partial r} + \Delta_2 \theta \frac{\partial u}{\partial \theta} \\
    &= \left( r_{xx} + r_{yy} \right) \frac{\partial u}{\partial r} + \left( \theta_{xx} + \theta_{yy} \right) \frac{\partial u}{\partial \theta} \\
    &= \left(
        \left( \cos \theta \right)_x + \left( \sin \theta \right)_y
    \right) \frac{\partial u}{\partial r}
    + \left(
        \left( - \frac{1}{r} \sin \theta \right)_x + \left( \frac{1}{r} \cos \theta \right)_y
    \right) \frac{\partial u}{\partial \theta} \\
    &= \left( 
        - \sin \theta * - \frac{1}{r} \sin \theta + \cos \theta * \frac{1}{r} \cos \theta
    \right) \frac{\partial u}{\partial r} + 
    \left( 
            \left( - \sin \theta \right)_x + \left( \cos \theta \right)_y
    \right) \frac{1}{r} \frac{\partial u}{\partial \theta} \\
    &\quad + \left( 
            - \sin \theta r_x + \cos \theta r_y
    \right) \frac{1}{r^2} \frac{\partial u}{\partial \theta} \\
    &=
    \frac{1}{r} \left( \sin^2 \theta + \cos^2 \theta \right) \frac{\partial u}{\partial r}  +
    \left( 
            - \cos \theta * - \frac{1}{r} \sin \theta + \left( - \sin \theta \right)  * \frac{1}{r} \cos \theta
    \right) \frac{1}{r} \frac{\partial u}{\partial \theta} \\
    &\quad + \left( 
            - \sin \theta \cos \theta + \cos \theta \sin \theta
    \right) \frac{1}{r^2} \frac{\partial u}{\partial \theta} \\
    &= \frac{1}{r} \frac{\partial u}{\partial r}
    ,
\end{align*}
$\Delta_2$ 为二阶微分算子, 故在极坐标变换下
\[
    \Delta_2 u
    = \left( \Delta_2 u \right)_1 + \left( \Delta_2 u \right)_2
    = \frac{\partial^2 u}{\partial r^2} + \frac{1}{r^2} \frac{\partial^2 u}{\partial \theta^2} + \frac{1}{r} \frac{\partial u}{\partial r}.
\]
\subsection{第 6 题}
设以 $P_0$ 为球心的半径为 $r$ 的球面为 $S_r$, 则
\begin{align*}
    \frac{3}{4 \pi R^3} \iiint_{K_R} u \, \mathrm{d} \Omega
    &= \frac{3}{4 \pi R^3} \int_{(0, R]} \int_{S_r} u \, \mathrm{d}S \, \mathrm{d}r \\
    &= \frac{3}{4 \pi R^3} \int_{(0, R]} 4 \pi r^2 \frac{1}{4 \pi r^2} \int_{S_r} u \, \mathrm{d}S \, \mathrm{d}r \\
    &= \frac{3}{R^3} \int_{(0, R]} r^2 \left( \frac{1}{4 \pi r^2} \int_{S_r} u \, \mathrm{d}S \right)  \, \mathrm{d}r
    ,
\end{align*}
由平均值定理可得
\[
    \frac{3}{4 \pi R^3} \iiint_{K_R} u \, \mathrm{d} \Omega
    = \frac{3}{R^3} \int_{(0, R]} r^2 u \left( P_0 \right) \, \mathrm{d}r
    = u \left( P_0 \right). 
\]

\end{document}
